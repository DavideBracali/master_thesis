\pagestyle{fancy}
\fancyhead{}

\fancyheadoffset{0cm}
\renewcommand{\headrulewidth}{1pt} 
\renewcommand{\footrulewidth}{0pt}
\fancyhead[L]{{\thesection}}
\fancyhead[L]{\leftmark}

% \fancyhead[LO,LE]{\nouppercase\firstleftmark}
%\fancyhead[RO,RE]{\thesubsection\enspace\subsectiontitle}

\section{Methods and Materials}


\subsection{Dataset Description}

The dataset for this project includes sEEG recordings, of the duration of
approximately 2 to 4 hours, for 23 patients affected by focal epilepsy.
Each patient provides one recording containing a seizure event and at least one
wakefulness segment.
For 18 patients two wakes were available, while the other 5 patients only
included one wake.

Each recording consists of several time series, one for each contact (typically
150-200), sampled at 500 Hz.
Each contact is labeled as a letter, representing the electrode, followed by an
integer number referring to a segment of that electrode.
For example, contact "A1" represents the first segment of electrode "A".

(!!! foto raw data??)


Beside the recordings a database was also provided, including the list of the
surgically removed brain regions.
Each of the patients in this study underwent surgery, consisting in the 
thermocoagulation of the tissue surrounding the contacts labeled as SOZ
(!!! assicurarsi di aver chiarito la sigla).
For all patients, the removal of these regions resulted in a drastic reduction
of the symptoms of epilepsy. We will therefore consider this list as the ground
truth for this project.
The onset time of the seizure, reported by doctors and corresponding to the
first electrical symptoms, was also included in the database.


\subsection{Pre-processing}

!! i miei appunti dicono
preprocessing:
notch 50 100 150 200 powerline
bandpass Butterworth 0.5-249.5 hz 6° ordine zero-phase (quindi in realtà 12° ordine)
moltiplicazione 1e6 prima di convertire a f32 (dati in uV)
chiedere ai miei supervisor ;) come hanno fatto

\subsection{The Brain Network}

Network analysis was extensively used in this work: sEEG recordings are
localized, and provide the opportunity to investigate the functional
relationships between different regions of the brain.
Each brain region can be mapped into a node, while an appropriate similarity
measure between the recordings can be used to form weighted links.
Because sEEG recordinggs are time series, such network will also be dynamic,
enabling us to gain insights on the evolution of the network before, during and
after the seizure.

This approach is widely explored in literature
\cite{10.1093/med/9780190228484.001.0001}, as epileptic seizures are
characterized by anomalous and synchronized electrical activity (!!! ciiiitare).
In particular, in focal epileptic seizures the abnormal activity is initially
limited between the SOZ, and subsequently spreads to the entirity (or a vast
area) of the brain.
Functional brain networks are powerful tools to model such synchronization and
diffusion mechanisms.

Different choices are possible for the similarity measure, depending on the
focus of the study \cite{10.1093/med/9780190228484.001.0001}.
For this work mutual information was chosen to quantify the dependencies of
signal pairs, capturing linear and non-linear relationships to model the
dynamics of an epileptic seizure. Mutual information between two discrete
random variables $X\in\mathcal{X}$ and $Y\in\mathcal{Y}$ is defined as:

\begin{equation}
    MI(X;Y) = D_{KL}(P_{(X,Y)}\ \Vert\ P_X \otimes P_Y)
\end{equation}

where $D_{KL}$ is the Kullback–Leibler divergence, a statistical distance
between the joint distribution $(P_{(X,Y)}$ and the outer product of the
marginal distributions $P_X$ and $P_Y$.
Notice 

\lipsum[2]

\subsection{Feature Extraction}
\lipsum[3]

\subsection{The Variational Autoencoder}
\lipsum[4]

\subsection{Gaussian Mixture Models}
\lipsum[5]

\subsection{Anomaly Detection}
\lipsum[6]

\subsection{Hyperparameter Tuning and Validation}
\lipsum[7]
