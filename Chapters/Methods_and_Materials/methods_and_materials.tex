\pagestyle{fancy}
\fancyhead{}

\fancyheadoffset{0cm}
\renewcommand{\headrulewidth}{1pt} 
\renewcommand{\footrulewidth}{0pt}
\fancyhead[L]{{\thesection}}
\fancyhead[L]{\leftmark}

% \fancyhead[LO,LE]{\nouppercase\firstleftmark}
%\fancyhead[RO,RE]{\thesubsection\enspace\subsectiontitle}

\section{Methods and Materials}


\subsection{Dataset Description}

The dataset for this project includes sEEG recordings, of the duration of
approximately 2 to 4 hours, for 23 patients affected by focal epilepsy.
Each patient provides one recording containing a seizure event and at least one
wakefulness segment.
For 18 patients two wakes were available, while the other 5 patients only
included one wake.

Each recording consists of several time series, one for each contact (typically
150-200), sampled at 500 Hz.
Each contact is labeled as a letter, representing the electrode, followed by an
integer number referring to a segment of that electrode.
For example, contact "A1" represents the first segment of electrode "A".

(!!! foto raw data??)


Beside the recordings a database was also provided, including the list of the
surgically removed brain regions.
Each of the patients in this study underwent surgery, consisting in the 
thermocoagulation of the tissue surrounding the contacts labeled as SOZ
(!!! assicurarsi di aver chiarito la sigla).
For all patients, the removal of these regions resulted in a drastic reduction
of the symptoms of epilepsy. We will therefore consider this list as the ground
truth for this project.
The onset time of the seizure, reported by doctors and corresponding to the
first electrical symptoms, was also included in the database.


\subsection{Pre-processing}

!! i miei appunti dicono
preprocessing:
notch 50 100 150 200 powerline
bandpass Butterworth 0.5-249.5 hz 6° ordine zero-phase (quindi in realtà 12° ordine)
moltiplicazione 1e6 prima di convertire a f32 (dati in uV)
chiedere ai miei supervisor ;) come hanno fatto

\subsection{The Brain Network} \label(meth:thebrainnetwork)

Network analysis was extensively used in this work: sEEG recordings are
localized, and provide the opportunity to investigate the functional
relationships between different regions of the brain.
Each brain region can be mapped into a node, while an appropriate similarity
measure between the recordings can be used to form weighted links.
Because sEEG recordinggs are time series, such network will also be dynamic,
enabling us to gain insights on the evolution of the network before, during and
after the seizure.

This approach is widely explored in literature
\cite{10.1093/med/9780190228484.001.0001}, as epileptic seizures are
characterized by anomalous and synchronized electrical activity (!!! ciiiitare).
In particular, in focal epileptic seizures the abnormal activity is initially
limited between the SOZ, and subsequently spreads to the entirity (or a vast
area) of the brain.
Functional brain networks are powerful tools to model such synchronization and
diffusion mechanisms.

Different choices are possible for the similarity measure, depending on the
focus of the study \cite{10.1093/med/9780190228484.001.0001}.
For this work mutual information was chosen to quantify the dependencies of
signal pairs, capturing linear and non-linear relationships to model the
dynamics of an epileptic seizure. Mutual information between two discrete
random variables $X\in\mathcal{X}$ and $Y\in\mathcal{Y}$ is defined as:

\begin{equation}
    MI(X;Y) = D_{KL}(P_{(X,Y)}\ \Vert\ P_X \otimes P_Y)
    = \sum_{x\in\mathcal(X)}\sum_{y\in\mathcal(Y)}
    {P_{(X,Y)}(x,y)\log{\frac{P_{(X,Y)}(x,y)}{P_{X}(x)\cdot P_{Y}(y)}}}
\end{equation}

where $D_{KL}$ is the Kullback–Leibler divergence, a statistical distance
between the joint distribution $(P_{(X,Y)}$ and the outer product of the
marginal distributions $P_X$ and $P_Y$. 
It is a non-negative and symmetric quantity that measures how much the joint
distribution of $(X,Y)$ differs from the product of the marginal distributions
of $X$ and $Y$ \cite{cover1991elements}. Hence, when two random variables
are independent, their mutual information is equal to zero.

Being a metric that quantifies the amount of shared information between two
variables, mutual information can be used to capture the linear and non-linear
relationships that model the synchronization dynamics typical of epilepsy.
For this reason, each sEEG channel was divided into $1\ s$ (or equivalently 500
time steps) non-overlapping time windows.
At a given instant, these segments represent the nodes of fully-connected,
weighted, undirected graph network.
The weight $w_{ij}$ of the link between two nodes $(i,j)$ is defined as
$w_{ij} = \frac{MI_{ij}}{MI_{ii}}$, where $MI_{ij}$ is the mutual information
between signal i and j.
It is normalized over the mutual information between identical signals so that
weights always are between 0 and 1, with $w_{ij} = 0$ indicating statistical
independence and $w_{ij} = 1$ corresponding to identical signals.

This process results in a dynamic set of networks with a temporal resolution of
$1\ s$. Higher temporal resolution could be possible, for example using a
sliding window instead of a non-overlapping one.
One must however keep in mind that, for each time frame, a different network has
to be constructed. The computation time to build and extract features will
therefore scale linearly with the number of time frames.

\subsection{Feature Extraction}

In order to distinguish between healthy and epileptogenic behavior in sEEG
recordings, a set of features must be computed at each time frame in order to
extract meaningful information from data.
In this study we decided to extract a total of 22 features, which can be grouped
into three categories depending on their nature: network features, signal
features and spectral features.

\subsubsection{Network features}

Network features are metrics that are computed from the functional network
described in section \ref(meth:thebrainnetwork).
Those quantities can both describe the global state of the brain network
(global features) and inspect the topological role of each individual node
within the network (local features).
Local metrics can provide insights on the individual role of each node in the
network.
Their main purpose in this study is to localize the portions of the brain who
play an important role during the seizure onset.
Global features and their evolution, on the other hand, can be important
to understand and analyze network dynamics before, during and after a seizure.

The following local measures were computed:

\subsubsection{Signal features}

\subsubsection{Spectral features}


\subsection{Variational Autoencoder (VAE)}


\subsection{Gaussian Mixture Models (GMM)}


\subsection{Negative log-likelihood}


\subsection{Classification}


\subsection{Hyperparameter Tuning}
