\pagestyle{fancy}
\fancyhead{}

\fancyheadoffset{0cm}
\renewcommand{\headrulewidth}{1pt} 
\renewcommand{\footrulewidth}{0pt}
\fancyhead[L]{{\thesection}}
\fancyhead[L]{\leftmark}

% \fancyhead[LO,LE]{\nouppercase\firstleftmark}
%\fancyhead[RO,RE]{\thesubsection\enspace\subsectiontitle}

\section{Methods and Materials}


\subsection{Dataset Description}

The dataset for this project includes sEEG recordings, of the duration of
approximately 2 to 4 hours, for 24 patients affected by focal epilepsy.
Each patient provides one recording containing a seizure event and at least one
wakefulness segment.
For 18 patients two wakes were avaiaible, while the other 6 patients only
included one wake.

Each recording consists of several time series, one for each contact (typically
150-200), sampled at 500 Hz.
Each contact is labeled as a letter, representing the electrode, followed by an
integer number referring to a segment of that electrode.
For example, contact "A1" represents the first segment of electrode "A".

A database was also included beside the recordings, including the
thermocoagulated segments for each patient, which we will consider as the ground
truth for this project. The onset time of the seizure, reported by doctors and
corresponding to the first electrical symptoms, was also included in the
database.

(!! foto raw data??)

\subsection{Pre-processing}

!! i miei appunti dicono
preprocessing:
notch 50 100 150 200 powerline
bandpass Butterworth 0.5-249.5 hz 6° ordine zero-phase (quindi in realtà 12° ordine)
moltiplicazione 1e6 prima di convertire a f32 (dati in uV)
chiedere ai miei supervisor ;) come hanno fatto

\subsection{The Brain Network}
\lipsum[2]

\subsection{Feature Extraction}
\lipsum[3]

\subsection{The Variational Autoencoder}
\lipsum[4]

\subsection{Gaussian Mixture Models}
\lipsum[5]

\subsection{Anomaly Detection}
\lipsum[6]

\subsection{Hyperparameter Tuning and Validation}
\lipsum[7]
