\pagestyle{fancy}
\fancyhead{}

\fancyheadoffset{0cm}
\renewcommand{\headrulewidth}{1pt} 
\renewcommand{\footrulewidth}{0pt}
\fancyhead[L]{{\thesection}}
\fancyhead[L]{\leftmark}

% \fancyhead[LO,LE]{\nouppercase\firstleftmark}
%\fancyhead[RO,RE]{\thesubsection\enspace\subsectiontitle}

\section{Introduction}

\subsection{Epilepsy}

\subsubsection{Generalized and Focal Epilepsy}

\subsubsection{Seizure Onset Zone (SOZ)}

\subsection{SOZ localization}

\subsubsection{Clinical relevance}

\subsubsection{Stereo-electroencephalography}

Stereoelectroencephalography (sEEG) is the practice of recording
electroencephalographic signals using surgically implanted electrodes.
Each electrode consists of multiple recording sites, which we will
refer as "contacts", capturing electrical activity from different depths within
the brain. This practice may be used on patients with epilepsy that do not
respond to medical treatment and who are potential candidates to receive brain
surgery in order to control their symptoms.

The strength of sEEG, when compared to standard electroencephalography (EEG),
is to be able to record electrical brain signals in deep 


\subsubsection{Problems and limitations}

\subsection{Thesis objective}
