\pagestyle{fancy}
\fancyhead{}

\fancyheadoffset{0cm}
\renewcommand{\headrulewidth}{1pt} 
\renewcommand{\footrulewidth}{0pt}
\fancyhead[L]{{\thesection}}
\fancyhead[L]{\leftmark}

% \fancyhead[LO,LE]{\nouppercase\firstleftmark}
%\fancyhead[RO,RE]{\thesubsection\enspace\subsectiontitle}

\section{Results and Discussion}

\subsection{Analysis on a subset of patients}

!!! Mostrare un paziente (se no diventa troppo lunga come immagini) come in
presentazione mettendo gli altri 3 in appendice

\subsubsection{Feature importance}\label{res:importance}

!!! Grafico a barre delle features, giustificare la scelta della betweenness come
peso per il classificatore

\subsubsection{Global seizure dynamics}

!!! Mostrare tutte e 22 le features, spiegando i trend più evidenti

\subsubsection{A classification example}

!!! Far vedere i plot dello spazio latente, della likelihood e 
grafico a barre del segnale d'allarme spiegando cosa succede

\subsection{Optimization Results}

!!! Per entrambi mostrare best params, curve di ottimizzazione, boxplot del target

\subsubsection{VAE and GMM}

\subsubsection{Alarm Classifier}

\subsection{Evaluation Metrics}

!!! Boxplots di pazienti validation e test

\subsection{Discussion}

\subsubsection{Validation and Test Results}

!!! Spiegare come mai il test viene molto peggio (pochi pazienti di training)
Confronto con random classifier e Bartolomei

\subsubsection{Hyperparameters Stability}

!!! Spiegare perché il training dipende fortemente dai pazienti e i best params
divergono (eterogeneità pazienti, distribuzioni latenti molto ampie)

\subsubsection{Future developments}

!!! Idee su come migliorare: fine tuning, VAE supervisionato, provare altri
modelli, classificatore deep learning